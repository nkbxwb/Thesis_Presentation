\begin{frame}{Causal Analysis}
Example: Examining weight loss between new weight loss drug or placebo
 \begin{itemize}
  \item We would like to be able to say ``The drug leads to more weight loss''
  \begin{itemize}
   \item But need an RCT to say this
   \begin{itemize}
    \item Randomization minimizes differences between groups at baseline
   \end{itemize}

   \item We only have observational data
   \item Thus differences could be attributed to the drug or confounding
   
   \begin{itemize}
    \item e.g. healthier people were much more likely to take the drug at baseline
   \end{itemize}

  \end{itemize}
Idea: Try to balance the covariates to reduce the effects of confounding
so the two groups seem identical at baseline
 \end{itemize}
\end{frame}

\begin{frame}{Counterfactuals}
\begin{itemize}
 \item Suppose that for or every person, there are two potential outcomes
 \begin{itemize}
  \item $Y_i(0)$ - The outcome if they had taken the control, $T=0$
  \item $Y_i(1)$ - The outcome if they had taken the treatment, $T=1$
 \end{itemize}
 \item The observed value for subject i: $Y_i=Y_i(1)T+Y_0(1-T)$
  \end{itemize}
\end{frame}




  
\begin{frame}{Counterfactual Example}
   \begin{figure}[h!]
  \centering
    \includegraphics[width=0.5\textwidth]{counterfactual.png}
    \caption{Example of a counterfactual}
\label{fig:counterfactual}
\end{figure}

\begin{itemize}
 \item Obviously we only observe one. \textit{The fundamental problem of causal inference}
\item If we could observe both, then we could observe the causal effects for each person
\end{itemize}
\end{frame}

\begin{frame}{Rubin's Causal Model}
\begin{itemize}
\item Stable unit treatment value assumption (SUTVA): Treatment status of another subject does not affect outcome of other units. Single version of each treatment
\item Ignorability/No Unmeasured Confounders: $(Y(0),Y(1))\perp T|X$
\end{itemize}
\note{No association between outcome and treatment assignment
The ignorability assumption simply means that the choice to assign to the control group
or the treatment group can be assumed to be effectively random when conditioned on observable 
characteristics of the study objects,}
\end{frame}




\begin{frame}{Estimands of Interest}
 \begin{itemize}
 \item Individual Treatment Effect: $Y_i(1)-Y_i(0)$
 \item Average Treatment Effect (ATE): $E[Y(1)-Y(0)]$. The effect of moving entire population
 from treated to untreated
 \item Average treatment effect for the treated (ATT): $E[Y(1)-Y(0)|T=1]$. The average treatment
 effect for those actually treated
 \item Note:  $E[Y(1)|T=1]\neq E[Y(1)]$, because $E[Y|T=1]=E[Y_1T+Y_0(1-T)|T=1]=E[Y_1|T=1]\neq E[Y(1)]$
\end{itemize}
 
\end{frame}


\begin{frame}{Propensity scores}
\begin{block}{Definition}
The propensity score is the probability that the subject received the treatment given the subjects \textit{pretreatment}
covariates. It is computed using the patient's baseline (pretreatment) information \cite{Rosenbaum1983}
\end{block}
 \begin{itemize}
  \item Defined as  $e_i(x)=P(T_i =1 |X_i)$
  \item Assume that the covariates play a role in how the subject chose treatment
  \item If we assume that $(Y(0),Y(1))\perp T|X \implies (Y(0),Y(1))\perp T|e(X)$, \cite{Rosenbaum1983}
  \item Controlling for propensity score will make groups seem indistinguishable
  \item Thus, we may treat it as if it were an RCT
 \end{itemize}

\end{frame}

\begin{frame}{Common Propensity Score Methods}
\begin{itemize}
 \item Matching: Match treatment and controls on their propensity score, calculate ATE
 \item Stratification: Stratify on propensity score, calculate ATE in each stratum
 \item Weighting: Weight each observation by the inverse of its propensity score, and then calculate ATE
 
  \begin{figure}[h!]
  \centering
    \includegraphics[width=0.8\textwidth]{ps_examples.png}
    \caption{Taken from TWANG short course \cite{Rand2015}}
\label{fig:psexamp}

\end{figure}
\end{itemize}

\note{Justification:
$E[ZY/e(x)]=E[ZY_1/e(x)]=E[E[ZY_1/e(X)]|Y_1,X]$
$=E[Y_1/e(x)E[Z|Y_1,X]]=E[Y_1/e(x) E[Z|X]]$
$=E[Y_1/e(x)e(x)]=E[Y_1]$
}
\end{frame}

%ps issues was here
\begin{frame}{IPTW}
\begin{itemize}
\item IPTW: Inverse probability of treatment weights
\item Idea: Weight sample by propensity score so that we get a sample where there is no confounding
\item Weights: $1/e(X)$ for treatment, $1/(1-e(X))$
\item Can be shown that $E[\frac{TY(1)}{e(X)}|T=1]=E[Y(1)]$ and $E[\frac{(1-T)Y(0)}{1-e(X)}|T=0]=E[Y(0)]$
\end{itemize}
\end{frame}
