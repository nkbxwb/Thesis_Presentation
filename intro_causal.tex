\begin{frame}{Causal Analysis}
Suppose we have a new drug we want to test to see how efficacious it is.
 \begin{itemize}
  \item We would like to be able to say ``The drug leads to better health''
  \begin{itemize}
   \item But need an RCT to say this
   \item We only have observational data
   \item Thus differences could be attributed to the drug or another factor (like healthier people
   decided to take the drug)
  \end{itemize}
Idea: try to balance the covariates so the two groups seem identical at baseline
 \end{itemize}

\end{frame}

\begin{frame}{Counterfactual Model}
\begin{itemize}
 \item Suppose that for or every person, there are two potential outcomes
 \begin{itemize}
  \item $Y_i(0)$ - The outcome if they had taken the control
  \item $Y_i(1)$ - The outcome if they had taken the treatment
 \end{itemize}
\item Obviously, we only observe one. The fundamental problem of causal inference
\item If we could observe both, then we could observe the causal effects for each person
\item We will have to settle for finding the average treatment effect (ATE)
\end{itemize}
 
\end{frame}

\begin{frame}{Propensity scores}
\begin{block}{Definition}
The propensity score is the probability that the subject received the treatment given the subjects
covariates. It is computed using the patient's baseline (pretreatment) information \cite{Rosenbaum1983}
\end{block}
 \begin{itemize}
  \item Assume that the covariates play a role in how the subject chose treatment
  \item Controling for propensity score will make groups seem indistinguishable
  \item Thus, we may treat it as if it were an RCT
 \end{itemize}

\end{frame}

\begin{frame}{Common Propensity Score Methods}
\begin{itemize}
 \item Matching: Match treatment and controls on their propensity score, calculate ATE
 \item Stratification: Stratify on propensity score, weight and combine ATE in each strate
 \item Weighting: Weight each observation by the inverse of its propensity score, and then calculate ATE
\end{itemize}

 
\end{frame}
