% Copyright 2004 by Till Tantau <tantau@users.sourceforge.net>.
%
% In principle, this file can be redistributed and/or modified under
% the terms of the GNU Public License, version 2.
%
% However, this file is supposed to be a template to be modified
% for your own needs. For this reason, if you use this file as a
% template and not specifically distribute it as part of a another
% package/program, I grant the extra permission to freely copy and
% modify this file as you see fit and even to delete this copyright
% notice. 


%%\documentclass[notes]{beamer}       % print frame + notes
%\documentclass[notes=only]{beamer}   % only notes
%\documentclass{beamer}              % only frames
\documentclass[xcolor=table]{beamer}
\setbeamertemplate{caption}[numbered]


%\documentclass{beamer}
\usepackage{lmodern}
\usepackage{verbatim}
\usepackage{graphicx}
\usepackage{adjustbox}
\usepackage{color,soul}
\usepackage{beamerthemesplit}
\usepackage{appendixnumberbeamer}
\usepackage{hhline}
%%%
%\usepackage[table]{xcolor} 
%\usepackage{booktabs}

\begin{comment}
 

%%%
\usepackage{xcolor,soul}
\definecolor{lightblue}{rgb}{.90,.95,1}
\sethlcolor{lightblue}
\renewcommand<>{\hl}[1]{\only#2{\beameroriginal{\hl}}{#1}}

% http://tex.stackexchange.com/questions/41683/why-is-it-that-coloring-in-soul-in-beamer-is-not-visible
\makeatletter
\newcommand\SoulColor{%
  \let\set@color\beamerorig@set@color
  \let\reset@color\beamerorig@reset@color}
\makeatother
\SoulColor
%%%
\end{comment}


% There are many different themes available for Beamer. A comprehensive
% list with examples is given here:
% http://deic.uab.es/~iblanes/beamer_gallery/index_by_theme.html
% You can uncomment the themes below if you would like to use a different
% one:
%\usetheme{AnnArbor}
%\usetheme{Antibes}
%\usetheme{Bergen}
%\usetheme{Berkeley}
%\usetheme{Berlin}
%\usetheme{Boadilla}
%\usetheme{boxes}
%\usetheme{CambridgeUS}
%\usetheme{Copenhagen}
%\usetheme{Darmstadt}
%\usetheme{default}
%\usetheme{Frankfurt}
%\usetheme{Goettingen}
%\usetheme{Hannover}
%\usetheme{Ilmenau}
%\usetheme{JuanLesPins}
%\usetheme{Luebeck}
\usetheme{Madrid}
%\usetheme{Malmoe}
%\usetheme{Marburg}
%\usetheme{Montpellier}
%\usetheme{PaloAlto}
%\usetheme{Pittsburgh}
%\usetheme{Rochester}
%\usetheme{Singapore}
%\usetheme{Szeged}
%\usetheme{Warsaw}

%%%%%

\makeatletter
\setbeamertemplate{footline}
{
  \leavevmode%
  \hbox{%
  \begin{beamercolorbox}[wd=.333333\paperwidth,ht=2.25ex,dp=1ex,center]{author in head/foot}%
    \usebeamerfont{author in head/foot}\insertsection
  \end{beamercolorbox}%
  \begin{beamercolorbox}[wd=.333333\paperwidth,ht=2.25ex,dp=1ex,center]{title in head/foot}%
    \usebeamerfont{title in head/foot}\insertsubsection
  \end{beamercolorbox}%
  \begin{beamercolorbox}[wd=.333333\paperwidth,ht=2.25ex,dp=1ex,right]{date in head/foot}%
    \usebeamerfont{date in head/foot}\insertshortdate{}\hspace*{2em}
    \insertframenumber{} / \inserttotalframenumber\hspace*{2ex} 
  \end{beamercolorbox}}%
  \vskip0pt%
}
\makeatother

%%%%

\title[Master's Thesis]{Using Multiple Imputation, Survival Analysis, 
And Propensity Score Analysis In Cancer Data With Missingness}

% A subtitle is optional and this may be deleted
\subtitle{Master's Thesis}

\author{Nathan Berliner}
% - Give the names in the same order as the appear in the paper.
% - Use the \inst{?} command only if the authors have different
%   affiliation.

\institute[Rice] % (optional, but mostly needed)
{
  %\inst{1}%
  Department of Statistics\\
  Rice University
  }

\date{11/30/2015}
% - Either use conference name or its abbreviation.
% - Not really informative to the audience, more for people (including
%   yourself) who are reading the slides online

\subject{Statistics}
% This is only inserted into the PDF information catalog. Can be left
% out. 

% If you have a file called "university-logo-filename.xxx", where xxx
% is a graphic format that can be processed by latex or pdflatex,
% resp., then you can add a logo as follows:

% \pgfdeclareimage[height=0.5cm]{university-logo}{university-logo-filename}
% \logo{\pgfuseimage{university-logo}}

% Delete this, if you do not want the table of contents to pop up at
% the beginning of each subsection:
\begin{comment}
 

\AtBeginSubsection[]
{
  \begin{frame}<beamer>{Outline}
    \tableofcontents[currentsection,currentsubsection]
  \end{frame}
}
\end{comment}

% Let's get started
\begin{document}

\begin{frame}
  \titlepage
\end{frame}

\begin{comment}
 
\begin{frame}{Outline}
  \tableofcontents
  % You might wish to add the option [pausesections]
\end{frame}


\end{comment}

% Section and subsections will appear in the presentation overview
% and table of contents.
\section{Introduction}
\subsection{The Problem}

%\subsection{}

\begin{frame}{In an ideal world}
  \begin{itemize}
  \item We would have a large dataset
  \begin{itemize}
   \item That was obtained from an RCT
   \item That would help answer a clearly defined question
   \item That had all the covariates of scientific interest
   \item That contained no missing data
  \end{itemize}

  \end{itemize}
\end{frame}

\begin{frame}{In Reality}

  \begin{itemize}
   \item RCT's are expensive and often unethical
   \begin{itemize}
    \item We often get retrospective observational data
    \item Pulled from a database or historical records
   \end{itemize}

   \item The questions we have may not be answerable from the data on hand
   \begin{itemize}
    \item The data obtained often doesn't support the original question in mind
   \end{itemize}

   \item The covariates collected are out of our control
   \begin{itemize}
    \item Since often no control of experiment, no control over what is collected
   \end{itemize}

   \item Lots of missing data
   \begin{itemize}
    \item Since no control over how the data is collected, we can't guarantee that everything is collected
   \item This issue is seemingly omnipresent in all types of data collection
   \end{itemize}

  \end{itemize}
  \note{hey!}

\end{frame}

\begin{frame}{Is This a Problem?}

  \begin{itemize}
   \item Without an RCT, we can't be sure if differences in treatments is due to the treatment or something else
   \item Omitting important factors may bias our results
   \item With missing data, we will be throwing away data and biasing our results
  \end{itemize}


\end{frame}

\begin{frame}{The Solution}
This thesis aims to fix some of these problems
  \begin{itemize}
   \item Fill in missing data via multiple imputation
   \item Create meaningful analytical models via survival analysis
   \item Get a causal interpretation from observational data
  \end{itemize}


\end{frame}

\begin{frame}{Motivation}
\begin{itemize}
   \item This thesis is motivated by cancer survival data with moderate missingness
   \item We will build the theory for dealing with this situation
   \item And then apply it to a cancer data set
  \end{itemize}


\end{frame}

\begin{frame}{Abstract}
In this thesis, multiple imputation, survival analysis, and propensity score analysis are combined in 
order to answer questions about cancer data with moderate missingness. While each of these fields have 
been studied individually, there has been little work and analysis on using the three in trio.
Starting with an incomplete dataset, we aim to impute the missing data, run survival analysis on each
of the imputed datasets, and then do propensity score analysis to observe causal effects.
Along the way, many theoretical and analytical decisions are made. I explain why each decision is made, 
and offer ample evidence for the other choices such that the interested reader may implement the methods
if they so choose. I apply the methodology to a cancer survival dataset in a case study, but the methods 
used are general, and could be adapted for any type of data.
 
\end{frame}


%\subsection{Missing data}
\begin{frame}{What is missing data}
 \begin{itemize}
 \item Missing data happens when we intend to collect a piece of data but don't actually get it
 \item Historical approaches
 \begin{itemize}
  \item Complete Case analysis: Throw away any record that is not complete
  % list of downsides
 \item Available Case analysis: Use records so long as they are complete for the specific analysis in question
 %bad things here
 \end{itemize}
 \end{itemize}
\end{frame}


\begin{frame}{Imputation}
\begin{block}{Definition}
The English verb ``to impute'' comes from the Latin imputo, which means
to reckon, attribute, make account of, charge, ascribe. \cite{VanBuuren2012}
\end{block}
\begin{itemize}
 \item In the 1930's, Allan, Wishart, and Yates laid framework for missing data
 \begin{itemize}
  \item Idea: Fill in the missing value, deduct degrees of freedom to account for it
  \item Issue: Dogmatic, and variance can't be estimates correctly
 \end{itemize}

\end{itemize}

 
\end{frame}

\begin{frame}{Multiple Imputation}
Throughout the 70's and 80's Donald Rubin worked to improve on this
\begin{itemize}
 \item Instead of imputing one value, lets impute it $m\geq 2$ times
 \item Draw the values from the missing data�s posterior distribution given the observed
 data and the process that generated the missing data
\end{itemize}
This idea is called Multiple Imputation (MI) and was formalized in 1987 \cite{Rubin1987}. It is the gold standard method
for missing data currently.
\end{frame}


\begin{frame}{How does MI work?}
 \begin{figure}[h!]
  \centering
    \includegraphics[width=0.8\textwidth]{mi_example_full.jpg}
  \caption{Visualization of MI data}
\label{fig:miexample}
\medskip
\small
Missingness is displayed by \textcolor{red}{?'s} and the imputed data is shown  as \textcolor{green}{\#'s}.
We then regress age on weight, get the results from the individual datasets, and then pool them together.
\end{figure}
\note{We will go in to much more detail later in presentation}
\end{frame}




%\subsection{Survival Analysis}
\begin{frame}{Survival Analysis}
\begin{block}{Survival Analysis}
Survival analysis is a field of statistics concerned with analyzing time to 
event data, often in the face of censoring or truncation.
\end{block}
Examples:
\begin{itemize}
 \item The survival of patients after a liver transplant in a hospital
 \begin{itemize}
  \item Complications: study ending, patients die before study starts, subject moves away
 \end{itemize}

 \item The time until a child learns a new task
 \begin{itemize}
  \item Complications: refuse participation, move away, don't recall the exact time they learned,
  already learned the task
 \end{itemize}

\end{itemize}
\end{frame}

\begin{frame}{Kaplan-Meier Estimator}
\begin{itemize}
 \item The survival function $S(t)=P(T>t)=\int_{t}^{\infty}f(u)du$ is estimated by the 
 nonparametric Kaplan-Meier Estimator
 $$\hat{S}(t)=\prod_{t_i<t}\frac{n_i -d_i}{n_i}$$
\item $n_i$ is the number of subject in the risk set at time $t_i$
\item $d_i$ is the number of deaths at time $t_i$
\end{itemize}
\note{
 risk set at time t; the set of individuals alive and uncensored just before time t.
We use death and survival because easy to say, but it really means event or not}
\end{frame}

\begin{frame}{Log rank test}
The log rank test compares two survival curves to see if from the same distribution

$$\frac{\sum_{j=1}^{J}w_j(O_{1j}-E_{1j})}{\sqrt{\sum_{j=1}^{j}w_j^2V_{j}}}\sim N(0,1)$$
\begin{itemize}
 \item Where $w_j$ is the weight of each observation (must be $\geq 0$, we will set all to be 1)
 \item $N_j=N_{1j}+N_{2j}$ is the number at risk at time j (composed from deaths in each group)
 \item $O_j=O_{1j}+O_{2j}$ is the observed number of deaths at time j (composed from the observed deaths in each group)
 \item $E_{1j}=\frac{O_jN_{1j}}{N_j}$
 \item $V_j=\frac{O_j(N_{1j}/N_j)(1-N_{1j}/N_j)(N_{j}-O_{j})}{N_j -1}$
 \end{itemize}
\note[itemize]{Weights such as

\item peto peto
\item gehan

will put emphasis on different parts of the survival curve
FIGURE THIS OUT
}
\end{frame}


\begin{frame}{Cox Regression}
%might want to use the underbraces
%$\underbrace{h_{0}(t)}_{\textrm{time}}*\underbrace{exp(\sum_{k=1}^{p}\beta_{k}Z_{k})}_{\textrm{covariates}}$
\begin{itemize}
 \item Hazard is the instantaneous rate of event given that you have survived until time t, given 
 by $$h(t)=\lim_{\Delta t \rightarrow 0+}\frac{P[t\leq T<t+\Delta t|T\geq t]}{\Delta t}$$
   \item Cox regression models hazard by 
   
   %$$h(t|Z)=h_{0}(t)\exp(\sum_{k=1}^{p}\beta_{k}Z_{k})$$
   $$h(t|Z)=\underbrace{h_{0}(t)}_{\textrm{time}}*\underbrace{exp(\sum_{k=1}^{p}\beta_{k}Z_{k})}_{\textrm{covariates}}$$

   \item Where $h_{0}(t)$ is the baseline hazard
   \item $Z_k$ is the $k^{th}$ covariate
   \item $\beta_k$'s are found by maximizing the partial likelihood function
 \end{itemize}
The covariates act to multiply the hazard function.
\end{frame}


%\subsection{Causal Analysis}
\begin{frame}{Causal Analysis}
Suppose we have a new drug we want to test to see how efficacious it is.
 \begin{itemize}
  \item We would like to be able to say ``The drug leads to better health''
  \begin{itemize}
   \item But need an RCT to say this
   \begin{itemize}
    \item Randomization minimizes differences between groups at baseline
   \end{itemize}

   \item We only have observational data
   \item Thus differences could be attributed to the drug or confounding
   
   \begin{itemize}
    \item e.g. healthier people were much more likely to take the drug at baseline
   \end{itemize}

  \end{itemize}
Idea: Try to balance the covariates to reduce the effects of confounding
so the two groups seem identical at baseline
 \end{itemize}

\end{frame}

\begin{frame}{Counterfactual Model}
\begin{itemize}
 \item Suppose that for or every person, there are two potential outcomes
 \begin{itemize}
  \item $Y_i(0)$ - The outcome if they had taken the control, $Z=0$
  \item $Y_i(1)$ - The outcome if they had taken the treatment, $Z=1$
 \end{itemize}
\item Obviously, we only observe one. \textit{The fundamental problem of causal inference}
\item If we could observe both, then we could observe the causal effects for each person
\item Estimands of interest:
\begin{itemize}
 \item Average Treatment Effect (ATE) $E[Y_i(1)-Y_i(0)]$. The effect of moving entire population
 from treated to untreated
 \item Average treatment effect for the treated (ATT) $E[Y_i(1)-Y_i(0)|Z=1]$. The average treatment
 effect for those actually treated
\end{itemize}
\item In survival, the ATE is the difference in survival time
\end{itemize}
 
\end{frame}

\begin{frame}{Propensity scores}
\begin{block}{Definition}
The propensity score is the probability that the subject received the treatment given the subjects \textit{pretreatment}
covariates. It is computed using the patient's baseline (pretreatment) information \cite{Rosenbaum1983}
\end{block}
 \begin{itemize}
  \item Defined as  $e_i(x)=P(Z_i =1 |X_i)$
  \item Assume that the covariates play a role in how the subject chose treatment
  \item If we assume that $(Y(0),Y(1))\perp T|X \implies (Y(0),Y(1))\perp T|e(X)$
  \item Controlling for propensity score will make groups seem indistinguishable
  \item Thus, we may treat it as if it were an RCT
 \end{itemize}

\end{frame}

\begin{frame}{Common Propensity Score Methods}
\begin{itemize}
 \item Matching: Match treatment and controls on their propensity score, calculate ATE
 \item Stratification: Stratify on propensity score, weight and combine ATE in each strata
 \item Weighting: Weight each observation by the inverse of its propensity score, and then calculate ATE
\end{itemize}
\end{frame}

\begin{frame}{Propensity Score Issues}
 \begin{itemize}
  \item Unmeasured confounders
  \item Choice of pretreatment covariates in the propensity score model
  \item Different models and methods may lead to different conclusions
 \end{itemize}

\end{frame}


%\section{Methods}
%\subsection{Imputation}

%\begin{frame}{A path with many options}
 \begin{itemize}
  \item There are many different options to choose
  \item I explain my choices but dicuss other options
  \item Goal: Be clear so other researchers can adapt my methodology to their problems
 \end{itemize}

\end{frame}

\begin{frame}{MI primer}
\begin{itemize}
 \item MI forms the base of this thesis
 \item There are lots of different ways to impute
 \item As long as we can impute valid imputations, we can analyze them
 \item Poor imputation leads to poor results (bias, variability, loss in power)
\end{itemize}
\end{frame}

\begin{frame}{MI Notation}
 
 \begin{itemize}
\item $Y$ is our whole dataset. It will have $i$ rows and $j$ columns. Some of the covariates in the dataset will be completely observed, and others will have missingness.
\item $Y_j$ is a specific column of Y. $Y_j$ is composed as $Y_j=(Y_{j,obs},Y_{j,mis})$, where
	\begin{itemize}
	\item $Y_{j,obs}$ is the data we have observed for covariate j
	\item $Y_{j,mis}$ is the missing data covariate j
\end{itemize} 
\item $Y_{obs}$ is all of the data that we have observed
\item $Y_{mis}$ is all the data that we have not observed
\item R is a binary matrix the same size as $Y$ where a 1 indicates we observed the data, and 0 means it is missing
\item $\psi$ is a vector of parameters for the missing data model. 
\item The missing data model is given as $p(R|Y_{obs},Y_{mis},\psi)$
\item $\theta$ is a vector of the parameters for the full model of $Y$
\end{itemize}
\end{frame}

\begin{frame}{MI Concepts}
\begin{itemize}
 \item Ignorability
$$p(Y_{mis}|Y_{obs},R)= p(Y_{mis}|Y_{obs})$$
That is, we may ``ignore'' the R. The probability of the data being missing does not depend on how the data is missing. 
Equivalently, we may write this as
$$p(Y_{mis}|Y_{obs},R=1)= p(Y_{mis}|Y_{obs},R=0)$$
\item Non ignorability: $$p(Y_{mis}|Y_{obs},R=1)\neq p(Y_{mis}|Y_{obs},R=0)$$
So we must take into account the missing data structure for imputation.


\end{itemize}

\note{Being ignorable makes it justified to model our missing data from our observed data, without needing to worry about how it was missing.
The opposite of ignorable data is called non-ignorable data, in this case.
We often times see ignorable missing data in practice,
although one should certainly check the sensibility of ignorability, as some instances will certainly be non-ignorable, 
for example censored data, or when we know that the missing data is systematically different than the observed. 
If we have strongly nonignorable data, we should either try one of two things.

The first is to expand the data (collect something else similar to the covariate with missingness) so that it becomes ignorable and 

the second is to formulate two imputation models, one for the observed and one for the missing.

}

\end{frame}

\begin{frame}{Missing data Mechanisms}
Now, we may discuss the three main types of missing data mechanisms. 
\begin{itemize}

\item MCAR: Missing completely at random:  $$P(R=0|Y_{obs},Y_{mis},\psi)=P(R=0|\psi)$$
\begin{itemize}
\item The missingness in the data is not at all related to any of the data that we do or don't have
\end{itemize}
\item MAR: Missing at random: $$p(R=0|Y_{obs},Y_{mis},\psi)= p(R=0|Y_{obs},\psi)$$
\begin{itemize}
 \item The missingness we have is related to something in the data 
\end{itemize}
\item MNAR: Missing not at random: $$p(R=0|Y_{obs},Y_{mis},\psi)$$ does not simplify
\begin{itemize}
 \item  and the missingness depends on data that we have as well as have not collected
\end{itemize}


\end{itemize}
\note{
 If a lab technician slips and drops 5 vials of blood, the missingness caused by this would be MCAR
 If we collect the gender of the subject and we know that males tend to not give blood, we can attribute the missingness to the gender. In general, MAR models are ignorable.
 For example if a full moon causes the blood testing machine to break more often, but we don't have the moon phase as a variable.
}
 
\end{frame}

\begin{frame}{Joint Modelling}
 \begin{itemize}
  \item Assume ignorable MAR  missing data mechanism
  \item Missing data imputed by sampling from a user specified distribution
  \item A lot of theory developed for Normal, not much else
  \begin{itemize}
   \item Normal imputation has been shown to perform well, even under non normality \cite{Demirtas2008}
  \end{itemize}
\item Idea: pull imputations by missing data row pattern
 \end{itemize}

\end{frame}

\begin{frame}{JM pseudocode}
 \begin{figure}[h!]
  \centering
    \includegraphics[width=0.6\textwidth]{jm_algo}
 % \caption{Normal JM imputation pseudocode}
\label{fig:jmexample}
\end{figure}
%do I want to include the amelia algo?
\end{frame}

\begin{frame}{JM Pros and Cons}
Pros
 \begin{itemize}
  \item Fast
  \item Easy to derive posteriors with common distributions
 \end{itemize}

 Cons
 \begin{itemize}
  \item Inflexible
  \item Limited to known distributions
  \item How to deal with mixed categorical and continous missing data
 \end{itemize}

\end{frame}

\begin{frame}{Full Conditional Specification}
 \begin{itemize}
  \item Assume MAR missing data mechanism %althouth MNAR with more 
  \item Missing data is imputed iteratively on a variable by variable basis
  \item Requires no distributional assumptions
  \item Idead: Specify k one dimensional models to impute on the missing data columns
 \end{itemize}

\end{frame}

%\subsection{Survival}
%\begin{frame}{Kaplan-Meier in the MI Setting}
 
\end{frame}


%\subsection{Causal Analysis}
%\begin{frame}{This needs a lot of work}
 
\end{frame}


%\section{Application}
%\subsection{Breast Brain Mets Example}

%\begin{frame}{Data Explanation}
\begin{itemize}
 \item 1514 MD Anderson patients who had brainmets from breast cancer
 \item 90 covariates
 \begin{itemize}
  \item Missingness from 0 to 65\%
 \end{itemize}

\end{itemize}
\begin{table}[!ht]
\centering
\begin{tabular}{|c|c|}
\hline
Type                                                                            & Example                                                                       \\ \hline
Subject data                                                                    & Age range, race, date of birth                                                \\ \hline
Cancer data                                                                     & TNM staging, type, receptor status                                            \\ \hline
\begin{tabular}[c]{@{}c@{}}Pre brain mets\\ data\end{tabular}                   & Treatment types                                                               \\ \hline
\begin{tabular}[c]{@{}c@{}}Post brain mets\\ clinical observations\end{tabular} & Seizures, headache, nasuea                                                    \\ \hline
\begin{tabular}[c]{@{}c@{}}Post brain mets\\ data\end{tabular}                  & \begin{tabular}[c]{@{}c@{}}Treatment type, \\ type of brain mets\end{tabular} \\ \hline
Survival data                                                                   & Survival time after brain mets, censoring indicator                           \\ \hline
\end{tabular}
\caption{Data Categories and Examples}
\label{table:cats}
\end{table}

\end{frame}

\begin{frame}{Important Covariates}
 \begin{table}[ !ht]
\centering
\adjustbox{max height=\dimexpr\textheight-5.5cm\relax,
           max width=\textwidth}{
\begin{tabular}{|c|c|c|}
\hline
Name        & \begin{tabular}[c]{@{}c@{}}Percent \\ Missing\end{tabular} & Meaning                                                                                                                                             \\ \hline
hrher2      & 5                                                        & \begin{tabular}[c]{@{}c@{}}Categorical variable: The hormonal receptor and \\ HER2 receptor status of the subject\end{tabular}                      \\ \hline
agebrainmet & 0                                                          & Indicator: Age greater or less than 60 at time of brain mets                                                                                        \\ \hline
timedx      & 1                                                         & \begin{tabular}[c]{@{}c@{}}Indicator: Time (years) from breast cancer diagnosis to brain\\ mets diagnosis greater or less than 6 years\end{tabular} \\ \hline
site5       & 1                                                        & Indicator: First metastasis was to brain                                                                                                            \\ \hline
race2       & 0                                                          & Categorical: White, Black, Hispanic, other                                                                                                          \\ \hline
priorn      & 0                                                          & \begin{tabular}[c]{@{}c@{}}Indicator: Number of prior treatments in metastatic setting \\ before brain mets\end{tabular}                            \\ \hline
braintype   & 4                                                        & Categorical: Single, multiple, Leptomeningeal disease                                                                                               \\ \hline
controlled  & 12                                                        & Indicator: Extracranial progression of brain mets                                                                                                   \\ \hline
capeothno   & 18                                                        & \begin{tabular}[c]{@{}c@{}}Indicator: Capecitabine, other, or no chemotheraputic\\ treatment. Treatment variable 1\end{tabular}                     \\ \hline
lapatrasno  & 18                                                        & \begin{tabular}[c]{@{}c@{}}Indicator: Lapatinib, Trastuzumab, or no HER2 treatment.\\ Treatment variable 2\end{tabular}                             \\ \hline
os          & 0                                                          & Overall survival (months)                                                                                                                           \\ \hline
dead        & 0                                                          & Indicator: death indicator                                                                                                                          \\ \hline
her2        & 10                                                        & Indicator: HER2 receptor status                                                                                                                     \\ \hline
\end{tabular}
}
\caption{Table of important covariates to be used in the analysis}
\label{table:importantvars}

\end{table}
\end{frame}

\begin{frame}{Visualization of Missingness}
 \begin{figure}[h!]
  \centering
    \includegraphics[width=0.8\textwidth]{missingvalues_plot.png}
  \caption{Visualization of missingness in the cancer dataset}
\label{fig:missingplot}

\end{figure}
\end{frame}

\begin{frame}{Imputation}
 \begin{itemize}
  \item MAR assumption seems reasonable
  \item FCS over JM due to nature of data
  \item Need to set up models and predictors
  \item Check for convergence and validity
 \end{itemize}

\end{frame}

\begin{frame}{Setting up the model}
Issues
\begin{itemize}
 \item Many categorical variables 
 \item Collinearity between predictors
 \item Variables with poor influx/outflux \cite{VanBuuren2012}
 \item How many iterations and imputations to draw?
\end{itemize}

 
\end{frame}


\begin{frame}{Convergence}
 
\includegraphics[width=.5\textwidth]{traceplots1}%
\includegraphics[width=.5\textwidth]{traceplots2} 

\note{On convergence, the different streams should be freely intermingled with one another,
without showing any definite trends. Convergence is diagnosed when the variance between different 
sequences is no larger than the variance within each individual sequence.}
\end{frame}

\begin{frame}{Validity}
 \begin{itemize}
  \item Lots of tools for continous imputations
  \item not many for categorical
  \begin{itemize}
   \item Solution: look at tables to verify validity
  \end{itemize}

 \end{itemize}

\end{frame}

\begin{frame}{Validity Checks}
\includegraphics[width=.5\textwidth]{cont_densplot}%
\includegraphics[width=.5\textwidth]{discrete_densplot} 
\end{frame}

\begin{frame}{Validity Checks}
 \includegraphics[width=.5\textwidth]{bw_timemet}%
\includegraphics[width=.5\textwidth]{bw_timedx} 
\end{frame}

\begin{frame}{Tabluar Checks}
 %figure this out
 %\includegraphics[width=.5\textwidth,height=.5\textwidth]{oscontrol_table}%
%\includegraphics[width=.5\textwidth]{oshrher2_table} 
\end{frame}

\begin{frame}{MI data Breakdown}
\begin{table}[!ht]
\adjustbox{max height=\dimexpr\textheight-5.5cm\relax,
           max width=\textwidth}{
\centering
\begin{tabular}{|r|c|c|c|c|}
\hline
\multicolumn{1}{|l|}{}                            & \begin{tabular}[c]{@{}c@{}}Sys therapy \\ available case\end{tabular} & \begin{tabular}[c]{@{}c@{}}Sys therapy \\ MI\end{tabular} & \begin{tabular}[c]{@{}c@{}}No Sys therapy \\ available case\end{tabular} & \begin{tabular}[c]{@{}c@{}}No Sys therapy \\ MI\end{tabular} \\ \hline
\multicolumn{1}{|l|}{Age (mean,sd)}               & 51.4(10.8)                                                            & 51.2(10.9)                                                & 52.7(11.9)                                                               & 52.9(11.4)                                                   \\ \hline
\multicolumn{1}{|l|}{Breast Cancer subtype}       &                                                                       &                                                           &                                                                          &                                                              \\ \hline
HR+/HER2-                                         & 27\%                                                                  & 31\%                                                      & 28\%                                                                     & 33\%                                                         \\ \hline
HR+/HER2+                                         & 19\%                                                                  & 18\%                                                      & 12\%                                                                     & 13\%                                                         \\ \hline
HR-/HER2+                                         & 22\%                                                                  & 20\%                                                      & 15\%                                                                     & 12\%                                                         \\ \hline
Triple negative                                   & 32\%                                                                  & 32\%                                                      & 45\%                                                                     & 42\%                                                         \\ \hline
\multicolumn{1}{|l|}{Prior therapies for stage 4} & 1(0-3)                                                                & 2(0-4)                                                    & 2(0-4)                                                                   & 2(0-4)                                                       \\ \hline
\multicolumn{1}{|l|}{Single brain lesion}         & 25\%                                                                  & 23\%                                                      & 23\%                                                                     & 20\%                                                         \\ \hline
\multicolumn{1}{|l|}{Controlled extra-cranial}    & 40\%                                                                  & 40\%                                                      & 35\%                                                                     & 36\%                                                         \\ \hline
\multicolumn{1}{|l|}{ECOG 0-1}                    & 84\%                                                                  & 70\%                                                      & 53\%                                                                     & 40\%                                                         \\ \hline
\multicolumn{1}{|l|}{Local Therapy}               &                                                                       &                                                           &                                                                          &                                                              \\ \hline
Resection Alone                                   & 5\%                                                                   & 5\%                                                       & 9\%                                                                      & 7\%                                                          \\ \hline
SBRT alone                                        & 13\%                                                                  & 12\%                                                      & 9\%                                                                      & 8\%                                                          \\ \hline
WBRT                                              & 60\%                                                                  & 59\%                                                      & 52\%                                                                     & 53\%                                                         \\ \hline
Resection/SBRT+WBRT                               & 12\%                                                                  & 14\%                                                      & 10\%                                                                     & 8\%                                                          \\ \hline
no local therapy                                  & 10\%                                                                  & 10\%                                                      & 20\%                                                                     & 23\%                                                         \\ \hline
\end{tabular}
}
\caption{Characteristics of available case data versus MI data}
\label{table:chartab}
\end{table}
 
\end{frame}

\begin{frame}{Kaplan-Meier in MI}
 \begin{itemize}
  \item Noninformative censoring reasonable
  \item Pooled by Rubin's Rules on Complimentary log-log
 \end{itemize}
 \begin{figure}[h!]
  \centering
\includegraphics[width=.8\textwidth]{cape_km}
\end{figure}
\end{frame}

\begin{frame}{Kaplan-Meier in MI}
 \begin{figure}[h!]
  \centering
\includegraphics[width=.8\textwidth]{lapat_km}
\end{figure}
 
\end{frame}

\begin{frame}{Log Rank Test}
\begin{table}[]
\centering
\begin{tabular}{|l|c|c|}
\hline
                & \multicolumn{2}{c|}{Chemo}                         \\ \hline
                & \multicolumn{1}{l|}{AC} & \multicolumn{1}{l|}{MI} \\ \hline
cape/other/none & \textless.0001          & \textless.0001          \\ \hline
cape/other      & 0.0321                  & 0.033                   \\ \hline
cape/none       & 0.00039                 & .0016                   \\ \hline
other/none      & \textless.0001          & \textless.0001          \\ \hline
\end{tabular}
\end{table}

\begin{table}[]
\centering
\begin{tabular}{|l|c|c|}
\hline
                   & \multicolumn{2}{c|}{HER2}                         \\ \hline
                   & \multicolumn{1}{l|}{AC} & \multicolumn{1}{l|}{MI} \\ \hline
Lapat/Traztuz/none & \textless.0001          & \textless.0001          \\ \hline
Lapat/Trastuz      & .87                     & .81                     \\ \hline
Lapta/none         & .00017                  & .00018                  \\ \hline
Trastuz/none       & \textless.0001          & \textless.0001          \\ \hline
\end{tabular}
\end{table}
\end{frame}



\section{Conclusion}

\begin{frame}{Recap}
 \begin{itemize}
  \item Applied survival and causal analysis on MI cancer data
  \item Found overall,  any treatments better than none
  \item Other chemotherapeutics better than Capecitabine
  \item Lapatinib and Trastuzumab are about the same
 \end{itemize}

\end{frame}


\begin{frame}{Critiques}
 \begin{itemize}
  \item MI skeptics and method critiques
  \item Use of propensity scores
  \item Assumptions made throughout
  \end{itemize}

\end{frame}

\subsection{Future}
\begin{frame}{Further Research and Extensions}

\begin{itemize}
 \item Exploring the ``other chemotherapeutics''
 \item Competing risks
 \item AFT models
 \item Differing propensity score methods and instrumental variables
 \item Estimating counterfactuals as an MI problem in MI setting
 
\end{itemize}

 
\end{frame}

\subsection{Thanks!}
\begin{frame}{Acknowledgments}
 \begin{itemize}
  \item My committee - For guiding me through the process
  \item Dr. Hess - For advising and mentoring me
  \item Dr. Ibrahim and Dr. Bugano - For letting me work with their data and advising
  on project matters
  \item Margaret Poon - For ALWAYS knowing what needs to be done
  \item My family - For unconditionally supporting me
  \item My friends � For believing in me and helping critique my thesis
  
 \end{itemize}

\end{frame}



\begin{frame}{Propensity Score Issues}
 \begin{itemize}
  \item Unmeasured confounders
  \item Choice of pretreatment covariates in the propensity score model
  \item Different models and methods may lead to different conclusions
 \end{itemize}

\end{frame}

\begin{frame}{Joint Modelling (JM)}
 \begin{itemize}
  \item Assume ignorable MAR  missing data mechanism
  \item Missing data imputed by sampling from a user specified distribution
  \item A lot of theory developed for Normal, not much else
  \begin{itemize}
   \item Normal imputation has been shown to perform well, even under non normality \cite{Demirtas2008}
  \end{itemize}
\item Idea: pull imputations by missing data row pattern
 \end{itemize}

\end{frame}

\begin{frame}{JM pseudocode}
 \begin{figure}[h!]
  \centering
    \includegraphics[width=0.6\textwidth]{jm_algo}
 % \caption{Normal JM imputation pseudocode}
\label{fig:jmexample}
\end{figure}
%do I want to include the amelia algo?
\end{frame}

\begin{frame}{JM Pros and Cons}
Pros
 \begin{itemize}
  \item Fast
  \item Easy to derive posteriors with common distributions
 \end{itemize}

 Cons
 \begin{itemize}
  \item Inflexible
  \item Limited to known distributions
  \item How to deal with mixed categorical and continuous missing data
 \end{itemize}

\end{frame}

\begin{frame}{The Stack Method}
 \begin{itemize}
  \item Rubin's Rules work well, but not always
  \begin{itemize}
   \item Ex: partitioning the MI data on an imputed variable
   \item Taking the average is not a good idea
  \end{itemize}
    \item Solution: Stack the MI datasets on top of each other to get one huge dataset
    \begin{itemize}
     \item Will get unbiased results
     \item But sample size is falsely inflated, thus cannot trust variance
    \end{itemize}
 \end{itemize}
 \begin{figure}[h!]
  \centering
    \includegraphics[width=0.6\textwidth]{stacked}
 % \caption{Normal JM imputation pseudocode}
\label{fig:stacked}
\end{figure}
\end{frame}

\begin{comment}
 

% You can reveal the parts of a slide one at a time
% with the \pause command:
\begin{frame}{Second Slide Title}
  \begin{itemize}
  \item {
    First item.
    \pause % The slide will pause after showing the first item
  }
  \item {   
    Second item.
  }
  % You can also specify when the content should appear
  % by using <n->:
  \item<3-> {
    Third item.
  }
  \item<4-> {
    Fourth item.
  }
  % or you can use the \uncover command to reveal general
  % content (not just \items):
  \item<5-> {
    Fifth item. \uncover<6->{Extra text in the fifth item.}
  }
  \end{itemize}
\end{frame}

\section{Second Main Section}

\subsection{Another Subsection}

\begin{frame}{Blocks}
\begin{block}{Block Title}
You can also highlight sections of your presentation in a block, with it's own title
\end{block}
\begin{theorem}
There are separate environments for theorems, examples, definitions and proofs.
\end{theorem}
\begin{example}
Here is an example of an example block.
\end{example}
\end{frame}

% Placing a * after \section means it will not show in the
% outline or table of contents.
\section*{Summary}

\begin{frame}{Summary}
  \begin{itemize}
  \item
    The \alert{first main message} of your talk in one or two lines.
  \item
    The \alert{second main message} of your talk in one or two lines.
  \item
    Perhaps a \alert{third message}, but not more than that.
  \end{itemize}
  
  \begin{itemize}
  \item
    Outlook
    \begin{itemize}
    \item
      Something you haven't solved.
    \item
      Something else you haven't solved.
    \end{itemize}
  \end{itemize}
\end{frame}



% All of the following is optional and typically not needed. 
\appendix
\section<presentation>*{\appendixname}
\subsection<presentation>*{For Further Reading}
\begin{comment}
 

\begin{frame}[allowframebreaks]
  \frametitle<presentation>{For Further Reading}
    
  \begin{thebibliography}{10}
    
  \beamertemplatebookbibitems
  % Start with overview books.

  \bibitem{Author1990}
    A.~Author.
    \newblock {\em Handbook of Everything}.
    \newblock Some Press, 1990.
 
    
  \beamertemplatearticlebibitems
  % Followed by interesting articles. Keep the list short. 

  \bibitem{Someone2000}
    S.~Someone.
    \newblock On this and that.
    \newblock {\em Journal of This and That}, 2(1):50--100,
    2000.
  \end{thebibliography}
\end{frame}
\end{comment}

\begin{frame}[allowframebreaks]
        \frametitle{References}
        \bibliographystyle{ieeetr}
	\bibliography{Research}
\end{frame}
\end{document}


