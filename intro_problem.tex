
%\subsection{}

\begin{frame}{In an ideal world}
  \begin{itemize}
  \item We would have a large dataset
  \begin{itemize}
   \item That was obtained from an RCT
   \item That would help answer a clearly defined question
   \item That had all the covariates of scientific interest
   \item That contained no missing data
  \end{itemize}

  \end{itemize}
\end{frame}

\begin{frame}{In Reality}

  \begin{itemize}
   \item RCT's are expensive and often unethical
   \begin{itemize}
    \item We often get retrospective observational data
    \item Pulled from a database or historical records
   \end{itemize}

   \item The questions we have may not be answerable from the data on hand
   \begin{itemize}
    \item The data obtained often doesn't support the original question in mind
   \end{itemize}

   \item The covariates collected are out of our control
   \begin{itemize}
    \item Since often no control of experiment, no control over what is collected
   \end{itemize}

   \item Lots of missing data
   \begin{itemize}
    \item Since no control over how the data is collected, we can't guarantee that everything is collected
   \item This issue is seemingly omnipresent in all types of data collection
   \end{itemize}

  \end{itemize}
  \note{hey!}

\end{frame}

\begin{frame}{Is This a Problem?}

  \begin{itemize}
   \item Without an RCT, we can't be sure if differences in treatments is due to the treatment or something else
   \item Omitting important factors may bias our results
   \item With missing data, we will be throwing away data and biasing our results
  \end{itemize}


\end{frame}

\begin{frame}{The Solution}
This thesis aims to fix some of these problems
  \begin{itemize}
   \item Fill in missing data via multiple imputation
   \item Create meaningful analytical models via survival analysis
   \item Get a causal interpretation from observational data
  \end{itemize}


\end{frame}

\begin{frame}{Motivation}
\begin{itemize}
   \item This thesis is motivated by cancer survival data with moderate missingness
   \item We will build the theory for dealing with this sitation
   \item And then apply it to a cancer data set
  \end{itemize}


\end{frame}

\begin{frame}{Abstract}
In this thesis, multiple imputation, survival analysis, and propensity score analysis are combined in 
order to answer questions about cancer data with moderate missingness. While each of these fields have 
been studied individually, there has been little work and analysis on using the three in trio.
Starting with an incomplete dataset, we aim to impute the missing data, run survival analysis on each
of the imputed datasets, and then do propensity score analysis to observe causal effects.
Along the way, many theoretical and analytical decisions are mode. I explain why each decision is made, 
and offer ample evidence for the other choices such that the interested reader may implement the methods
if they so choose. I apply the methodology to a cancer survival dataset in a case study, but the methods 
used are general, and could be adapted for any type of data.
 
\end{frame}
