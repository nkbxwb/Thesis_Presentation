
%\subsection{}

\begin{frame}{In an ideal world}
  \begin{itemize}
  \item We would have a large dataset
  \begin{itemize}
   \item That was obtained from a randomized controlled trial (RCT)
   \item That would help answer a clearly defined question
   \item That had all the covariates of scientific interest
   \item That contained no missing data
  \end{itemize}

  \end{itemize}
\end{frame}

\begin{comment}
 
 %took out ``in reality'' slide
\begin{frame}{In Reality}

  \begin{itemize}
   \item RCT's are expensive and often unethical
   \begin{itemize}
    \item We often get retrospective observational data
    %\item Pulled from a database or historical records
   \end{itemize}

   \item The questions we have may not be answerable from the data on hand
   \begin{itemize}
    \item The data obtained often doesn't support the original question in mind
   \end{itemize}

   \item The covariates collected are out of our control
   \begin{itemize}
    \item Since often no control of experiment, no control over what is collected
   \end{itemize}

   \item Lots of missing data
   \begin{itemize}
    \item Since no control over how the data is collected, we can't guarantee that everything is collected
  % \item This issue is seemingly omnipresent in all types of data collection
   \end{itemize}

  \end{itemize}
  \note{hey!}

\end{frame}

\end{comment}

\begin{frame}{Is This a Problem?}

\begin{center}
 {\Huge{YES!}}
\end{center}
  \begin{itemize}
   \item Without an RCT, we can't be sure if differences in outcome is due to the treatment or something else
   \item Omitting important factors may bias our results
   \item With missing data, we will be throwing away data and biasing our results
  \end{itemize}


\end{frame}

\begin{frame}{The Solution}
This thesis aims to fix some of these problems
  \begin{itemize}
   \item Fill in missing data via multiple imputation
   \item Create meaningful analytical models via survival analysis
   \item Get a causal interpretation from observational data
  \end{itemize}
Goal: To be able to apply methods to cancer data

\end{frame}


\begin{frame}{Data Explanation}
\begin{itemize}
 \item 1514 MD Anderson patients who had brain mets from breast cancer between October 2009 and 
 December 2012
 \item 1242 usable cases
 \item 90 covariates
 \begin{itemize}
  \item Missingness from 0 to 65\%
 \end{itemize}

\end{itemize}
\begin{table}[!ht]
\centering
\begin{tabular}{|c|c|}
\hline
Type                                                                            & Example                                                                       \\ \hline
Subject data                                                                    & Age range, race, date of birth                                                \\ \hline
Breast Cancer data                                                                     & TNM staging, type, receptor status                                            \\ \hline
\begin{tabular}[c]{@{}c@{}}Pre brain mets\\ data\end{tabular}                   & Treatment types                                                               \\ \hline
\begin{tabular}[c]{@{}c@{}}Post brain mets\\ clinical observations\end{tabular} & Seizures, headache, nausea                                                    \\ \hline
\begin{tabular}[c]{@{}c@{}}Post brain mets\\ data\end{tabular}                  & \begin{tabular}[c]{@{}c@{}}Treatment type, \\ type of brain mets\end{tabular} \\ \hline
Survival data                                                                   & Survival time after brain mets, censoring indicator                           \\ \hline
\end{tabular}
\caption{Data Categories and Examples}
\label{table:datacats}
\end{table}

\end{frame}

\begin{frame}{Questions of interest}
Want to explore...
\begin{enumerate}
 \item Chemotherapeutic drugs: Capecitabine vs other chemotherapeutic agents
 \item HER2 directed therapies (Lapatinib, Trastuzumab) in HER2+ subjects \\~\\
 %  might want to move this later
 \end{enumerate}
 
 
 Note: treatment not determined at time of diagnosis 
 \begin{itemize}
  \item landmark (2 months) 
 \end{itemize}
 
\note{Human epidermal growth factor receptor 2 (HER2) overexpression drives the biology of 20
 pct of breast cancers, and predicts a poor prognosis for patients.}

\end{frame}

\begin{frame}{Plan For This Presentation}
Will put a graphic here of the flow of the paper
 
\end{frame}


\begin{comment}
 

\begin{frame}{Motivation}
\begin{itemize}
   \item This thesis is motivated by cancer survival data with moderate missingness
   \item We will build the theory for dealing with this situation
   \item And then apply it to a cancer data set
  \end{itemize}


\end{frame}

\begin{frame}{Abstract}
In this thesis, multiple imputation, survival analysis, and propensity score analysis are combined in 
order to answer questions about cancer data with moderate missingness. While each of these fields have 
been studied individually, there has been little work and analysis on using the three in trio.
Starting with an incomplete dataset, we aim to impute the missing data, run survival analysis on each
of the imputed datasets, and then do propensity score analysis to observe causal effects.
Along the way, many theoretical and analytical decisions are made. I explain why each decision is made, 
and offer ample evidence for the other choices such that the interested reader may implement the methods
if they so choose. I apply the methodology to a cancer survival dataset in a case study, but the methods 
used are general, and could be adapted for any type of data.
 
\end{frame}
\end{comment}
